\documentclass{article}

\usepackage{graphicx}

\usepackage{color}

%--- Encoding and language
%\usepackage[utf8]{inputenc}
\usepackage[T1]{fontenc}
\usepackage[english]{babel}
\usepackage{csquotes}

%--- Links
\usepackage[colorlinks=true,
            filecolor=green,
            menucolor=blue,
            linkcolor=blue,
            runcolor=green,
            urlcolor=green,
            anchorcolor=black,
            citecolor=red]
            {hyperref}

\usepackage[]{cleveref}

\newcommand{\todo}[1]{\textcolor{red}{\textbf{TODO}: #1}}

\begin{document}

\section{Runnable tests}

DLBC aims to have an extensive suite of runnable tests. When executed, the output of the test will be compared to a reference dataset.
To aid in managing tests, the script \textsc{./tests/runnable/run-tests.py} has been created.
Currently the script handles the following options:

\input{generated/run-tests.tex}

\noindent
Tests are specified through \textsc{.json} files. An example is shown below:

\input{generated/example-json.tex}

\begin{itemize}
\item \textbf{name} (required): Short name of the test. Usually this will be the same as the directory name.
\item \textbf{description} (required): Short description of the test. Usually one line.
\item \textbf{tags} (optional): List of tags that give information about which subsystems of the code are being tested.
\item \textbf{latex} (optional): Path to a file containing a LaTeX formatted (long) description of the test. The token \textsc{\%path\%} in the LaTeX file can be used to insert the path for plots; it will be replaced by the path to the reference data of the test.
\item \textbf{plot} (optional): List of scripts to execute to generate plots. If \textsc{run-tests.py} is called with \textsc{--plot}, all tests are first executed, and the plots are based on the new data. If \textsc{--plot-reference} is used instead, the tests are not run, but the reference data is plotted.
\item \textbf{configuration} (required): Version of DLBC to use; d3q19, d2q9, d1q5 or d1q3.
\item \textbf{input-file} (required): Base input file to use. Ususally this will be the same as the directory name, plus \textsc{.in}.
\item \textbf{np} (optional, default value is 1): Number of ranks to run the test on. This is overruled by the \textsc{parallel.nc} parameter, if specified.
\item \textbf{clean} (optional): List of paths to remove when cleaning the test.
\item \textbf{parameters} (optional): List of parameter names and values to be passed to DLBC. All combinations of values will be run, e.g. in this example the following will be passed:
\begin{itemize}
\item \textsc{--parameter ``some.par=val1'' --parameter ``another.par=1''}
\item \textsc{--parameter ``some.par=val1'' --parameter ``another.par=2''}
\item \textsc{--parameter ``some.par=val1'' --parameter ``another.par=42''}
\item \textsc{--parameter ``some.par=val2'' --parameter ``another.par=1''}
\item \textsc{--parameter ``some.par=val2'' --parameter ``another.par=2''}
\item \textsc{--parameter ``some.par=val2'' --parameter ``another.par=42''}
\end{itemize}
If \textsc{--only-first} is passed to \textsc{./run-tests.py}, only the first combination will be tested. This may be useful when doing a quick test. If no parameters are specified, nothing is passed to DLBC.
\item \textbf{compare} (required): How to compare the generated data to the reference data. Three values are currently used:
\begin{itemize}
\item \textbf{data} (required): Types of data to be compared. These will normally be the prefixes of the output files.
\item \textbf{comparision} (required): An array of comparison operations to be run. The value \textsc{type} specifies which comparision command to invoke. Currently, only \textsc{h5diff} is supported. The value \textsc{files} then are used for testing. Some tokens are supported, denoted \textsc{\%token\%}: all parameters specified are available as tokens, as well as \textsc{data}, which takes its values from the data array described above, and \textsc{np}, which takes its value from the \textsc{np} value, if specified, or the product of the values in the \textsc{parallel.nc} parameter, if specified for testing.
\item \textbf{shell} (optional): Extra shell commands to be executed. These should return a 0 exit code on success and non-zero on failure.
\end{itemize}
\end{itemize}


\subsection{List of tests}

\input{generated/list-of-tests.tex}


\end{document}

